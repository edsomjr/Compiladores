\section{Tabela de símbolos}

\begin{frame}[fragile]{Tabela de símbolos}

    \begin{itemize}
        \item Uma tabela de símbolos é uma estrutura de dados que armazena informações nas diferentes fases da compilação
       %\pause

        \item As fases de análise coletam informações que são usadas nas fases de síntese para a geração do programa alvo
       %\pause

        \item Por exemplo, na análise léxica os lexemas são adicionados à tabela de símbolos
       %\pause

        \item As fases posteriores podem adicionar informações a este lexema, como tipo, uso (procedimento, variável, etc) e posição no armazenamento
    \end{itemize}

\end{frame}

\begin{frame}[fragile]{A análise léxica e a tabela de símbolos}

    \begin{itemize}
        \item As principais interações que ocorrem com a tabela de símbolos na análise léxica tratam do armazenamento e recuperação de lexemas
       %\pause

        \item A rotina \Call{inserir}{$s, t$} insere o lexema $s$ na tabela, sendo ele associado ao token $t$
       %\pause

        \item A rotina \Call{buscar}{$s$} regasta o lexema $s$, permitindo a consulta (e, se necessário, atualização) de seu token
       %\pause

        \item Em geral a operação de busca é usada para saber se um determinado lexema já está na tabela ou não
       %\pause

        \item Em linguagem que possuem dicionários em sua biblioteca padrão, a tabela pode ser implementada por meio de um dicionário
    \end{itemize}

\end{frame}

\begin{frame}[fragile]{Tabela de símbolos e palavras reservadas}

    \begin{itemize}
        \item As rotinas descritas para a tabela de símbolos permitem um tratamento direto de quaisquer palavras reservadas da linguagem
       %\pause

        \item Por exemplo, considere que \textbf{div} e \textbf{mod} são palavras reservadas
       %\pause

        \item A rotina de inserção permite associar os lexemas \code{cpp}{"div"} e \code{cpp}{"mod"} aos tokens \textbf{div} e \textbf{mod}:
        \begin{center}
            \begin{tabular}{l}
                \Call{inserir}{\code{cpp}{"div"}, \textbf{div}} \\
                \Call{inserir}{\code{cpp}{"mod"}, \textbf{mod}} \\
            \end{tabular}
        \end{center}
       %\pause

        \item Qualquer chamada posterior da rotina de busca retornará os tokens, de modo que os lexemas já não poderão ser mais usados como identificadores
    \end{itemize}

\end{frame}

\begin{frame}[fragile]{Pseudocódigo de um analisador léxico que manipula identificadores}

    \begin{algorithmic}[1]
        \Function{scanner}{\,}
        \Loop
            \State{$c \gets \Call{proximoCaractere}{\,}$}
            \If{$c$ é espaço}
                \State{reinicie o laço}
            \ElsIf{$c$ é um dígito}
                \State{$v \gets$ \Call{lerConstanteInteira}{\,}}
                \State{\Return{\{ \code{cpp}{NUM}, $v$} \}}
            \ElsIf{$c$ é uma letra}
                \State{$lexema \gets$ \Call{lerPalavra}{\,}}
                \If{$lexema$ não está na tabela de símbolos}
                    \State{\Call{inserir}{\code{cpp}{ID}, $lexema$}}
                \EndIf
                \State{\Return{\{ \Call{buscar}{$lexema$}, $lexema$} \}}
            \Else
                \State{\Return{\{ $c$, \code{cpp}{NONE}} \}}
            \EndIf
        \EndLoop
        \EndFunction
    \end{algorithmic}

\end{frame}

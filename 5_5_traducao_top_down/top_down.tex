\section{Tradução {\it top-down}}

\begin{frame}[fragile]{Esquemas de tradução e recursão à esquerda}

    \begin{itemize}
        \item Na prática, trabalhar com um esquema de tradução, ao invés de uma definição dirigida pela sintaxe, é vantajoso no sentido de que, em um esquema
            de tradução, a ordem de avaliação das ações semânticas e dos valores dos atributos é dada explicitamente
       %\pause

        \item Como a maioria dos operadores aritméticos são associativos à esquerda, o uso de gramaticas recursivas à esquerda acaba sendo uma consequência
            natural
       %\pause

        \item Eliminar a recursão à esquerda permite a implementação de um tradutor preditivo
       %\pause

        \item Contudo, a eliminação à esquerda, sem o devido cuidado, pode modificar a associatividade do operador e, consequentemente, alterar o resultado das 
        expressões
       %\pause

        \item Compare as duas expressões abaixo, onde cada linha com uma associatividade diferente
            \inputsyntax{apl}{codes/expr.apl} 
    \end{itemize}

\end{frame}

\begin{frame}[fragile]{Esquema de tradução com uma gramática recursiva à esquerda}

\[
    \begin{array}{rlp{2cm}l}
        E \to& E_1 + T & & \{E.val := E_1.val + T.val\} \\
        E \to& E_1 - T & & \{E.val := E_1.val - T.val\} \\
        E \to& T & & \{E.val := T.val\} \\
        T \to& (E) & & \{T.val := E.val\} \\
        T \to& \textbf{num} & & \{T.val := \textbf{num}.val\}
    \end{array}
\]

\end{frame}

\begin{frame}[fragile]{Esquema de tradução transformado com uma gramática recursiva à direita}

\[
    \begin{array}{rlp{2cm}l}
        E \to& T & & \{R.h := T.val\} \\
             & R & & \{E.val := R.s\} \\
        R \to& + & & \\
             & T & & \{R_1.h := R.h + T.val\} \\
             & R_1 & & \{R.s := R_1.s\} \\
        R \to& - & & \\
             & T & & \{R_1.h := R.h - T.val\} \\
             & R_1 & & \{R.s := R_1.s\} \\
        R \to& \code{apl}{∊} & & \{R.s := R.h\} \\
        T \to& ( & & \\
             & E & & \\
             & ) & & \{T.val := E.val\} \\
        T \to& \textbf{num} & & \{T.val := \textbf{num}.val\}\\
    \end{array}
\]
\end{frame}

\begin{frame}[fragile]{Avaliação da expressão \code{apl}{1-2+3}}

    \begin{tikzpicture}
        \node[opacity=0] at (0, 0) { . }; 
        \node[opacity=0] at (14, 7) { . }; 

        \node (E) at (4, 6) { $E$ };
    \end{tikzpicture}

\end{frame}

\begin{frame}[fragile]{Avaliação da expressão \code{apl}{1-2+3}}

    \begin{tikzpicture}
        \node[opacity=0] at (0, 0) { . }; 
        \node[opacity=0] at (14, 7) { . }; 

        \node (E) at (4, 6) { $E$ };
        \node (T) at (1, 4.75) { $T$ };
        \node (R) at (7, 4.75) { $R$ };

        \draw[thick,dotted] (E) to (T);
        \draw[thick,dotted] (E) to (R);
    \end{tikzpicture}

\end{frame}

\begin{frame}[fragile]{Avaliação da expressão \code{apl}{1-2+3}}

    \begin{tikzpicture}
        \node[opacity=0] at (0, 0) { . }; 
        \node[opacity=0] at (14, 7) { . }; 

        \node (E) at (4, 6) { $E$ };
        \node (T) at (1, 4.75) { $T$ };
        \node (R) at (7, 4.75) { $R$ };
        \node (N1) at (1, 3.5) { $\textbf{num}$ };

        \draw[thick,dotted] (E) to (T);
        \draw[thick,dotted] (E) to (R);
        \draw[thick,dotted] (N1) to (T);
    \end{tikzpicture}

\end{frame}

\begin{frame}[fragile]{Avaliação da expressão \code{apl}{1-2+3}}

    \begin{tikzpicture}
        \node[opacity=0] at (0, 0) { . }; 
        \node[opacity=0] at (14, 7) { . }; 

        \node (E) at (4, 6) { $E$ };
        \node (T) at (1, 4.75) { $T$ };
        \node (R) at (7, 4.75) { $R$ };
        \node (N1) at (1, 3.5) { $\textbf{num}.val = 1$ };

        \draw[thick,dotted] (E) to (T);
        \draw[thick,dotted] (E) to (R);
        \draw[thick,dotted] (N1) to (T);
    \end{tikzpicture}

\end{frame}

\begin{frame}[fragile]{Avaliação da expressão \code{apl}{1-2+3}}

    \begin{tikzpicture}
        \node[opacity=0] at (0, 0) { . }; 
        \node[opacity=0] at (14, 7) { . }; 

        \node (E) at (4, 6) { $E$ };
        \node (T) at (1, 4.75) { $T.val = 1$ };
        \node (R) at (7, 4.75) { $R$ };
        \node (N1) at (1, 3.5) { $\textbf{num}.val = 1$ };

        \draw[thick,dotted] (E) to (T);
        \draw[thick,dotted] (E) to (R);
        \draw[thick,dotted] (N1) to (T);
        \draw[thick,-latex] (N1) to [bend left] (T);
    \end{tikzpicture}

\end{frame}

\begin{frame}[fragile]{Avaliação da expressão \code{apl}{1-2+3}}

    \begin{tikzpicture}
        \node[opacity=0] at (0, 0) { . }; 
        \node[opacity=0] at (14, 7) { . }; 

        \node (E) at (4, 6) { $E$ };
        \node (T) at (1, 4.75) { $T.val = 1$ };
        \node (R) at (7, 4.75) { $R.h = 1$ };
        \node (N1) at (1, 3.5) { $\textbf{num}.val = 1$ };

        \draw[thick,dotted] (E) to (T);
        \draw[thick,dotted] (E) to (R);
        \draw[thick,dotted] (N1) to (T);
        \draw[thick,-latex] (N1) to [bend left] (T);
        \draw[thick,-latex] (T) to (R);
    \end{tikzpicture}

\end{frame}

\begin{frame}[fragile]{Avaliação da expressão \code{apl}{1-2+3}}

    \begin{tikzpicture}
        \node[opacity=0] at (0, 0) { . }; 
        \node[opacity=0] at (14, 7) { . }; 

        \node (E) at (4, 6) { $E$ };
        \node (T) at (1, 4.75) { $T.val = 1$ };
        \node (R) at (7, 4.75) { $R.h = 1$ };
        \node (N1) at (1, 3.5) { $\textbf{num}.val = 1$ };

        \node (M) at (5.5, 3.5) { $-$ };
        \node (T2) at (7, 3.5) { $T$ };
        \node (R2) at (10, 3.5) { $R$ };

        \draw[thick,dotted] (E) to (T);
        \draw[thick,dotted] (E) to (R);
        \draw[thick,dotted] (N1) to (T);
        \draw[thick,dotted] (M) to (R);
        \draw[thick,dotted] (T2) to (R);
        \draw[thick,dotted] (R2) to (R);
        \draw[thick,-latex] (N1) to [bend left] (T);
        \draw[thick,-latex] (T) to (R);
    \end{tikzpicture}

\end{frame}

\begin{frame}[fragile]{Avaliação da expressão \code{apl}{1-2+3}}

    \begin{tikzpicture}
        \node[opacity=0] at (0, 0) { . }; 
        \node[opacity=0] at (14, 7) { . }; 

        \node (E) at (4, 6) { $E$ };
        \node (T) at (1, 4.75) { $T.val = 1$ };
        \node (R) at (7, 4.75) { $R.h = 1$ };
        \node (N1) at (1, 3.5) { $\textbf{num}.val = 1$ };

        \node (M) at (5.5, 3.5) { $-$ };
        \node (T2) at (7, 3.5) { $T$ };
        \node (R2) at (10, 3.5) { $R$ };
        \node (N2) at (7, 2.25) { $\textbf{num}$ };

        \draw[thick,dotted] (E) to (T);
        \draw[thick,dotted] (E) to (R);
        \draw[thick,dotted] (N1) to (T);
        \draw[thick,dotted] (M) to (R);
        \draw[thick,dotted] (T2) to (R);
        \draw[thick,dotted] (R2) to (R);
        \draw[thick,dotted] (N2) to (T2);
        \draw[thick,-latex] (N1) to [bend left] (T);
        \draw[thick,-latex] (T) to (R);
    \end{tikzpicture}

\end{frame}

\begin{frame}[fragile]{Avaliação da expressão \code{apl}{1-2+3}}

    \begin{tikzpicture}
        \node[opacity=0] at (0, 0) { . }; 
        \node[opacity=0] at (14, 7) { . }; 

        \node (E) at (4, 6) { $E$ };
        \node (T) at (1, 4.75) { $T.val = 1$ };
        \node (R) at (7, 4.75) { $R.h = 1$ };
        \node (N1) at (1, 3.5) { $\textbf{num}.val = 1$ };

        \node (M) at (5.5, 3.5) { $-$ };
        \node (T2) at (7, 3.5) { $T$ };
        \node (R2) at (10, 3.5) { $R$ };
        \node (N2) at (7, 2.25) { $\textbf{num}.val = 2$ };

        \draw[thick,dotted] (E) to (T);
        \draw[thick,dotted] (E) to (R);
        \draw[thick,dotted] (N1) to (T);
        \draw[thick,dotted] (M) to (R);
        \draw[thick,dotted] (T2) to (R);
        \draw[thick,dotted] (R2) to (R);
        \draw[thick,dotted] (N2) to (T2);
        \draw[thick,-latex] (N1) to [bend left] (T);
        \draw[thick,-latex] (T) to (R);
    \end{tikzpicture}

\end{frame}

\begin{frame}[fragile]{Avaliação da expressão \code{apl}{1-2+3}}

    \begin{tikzpicture}
        \node[opacity=0] at (0, 0) { . }; 
        \node[opacity=0] at (14, 7) { . }; 

        \node (E) at (4, 6) { $E$ };
        \node (T) at (1, 4.75) { $T.val = 1$ };
        \node (R) at (7, 4.75) { $R.h = 1$ };
        \node (N1) at (1, 3.5) { $\textbf{num}.val = 1$ };

        \node (M) at (5.5, 3.5) { $-$ };
        \node (T2) at (7, 3.5) { $T.val = 2$ };
        \node (R2) at (10, 3.5) { $R$ };
        \node (N2) at (7, 2.25) { $\textbf{num}.val = 2$ };

        \draw[thick,dotted] (E) to (T);
        \draw[thick,dotted] (E) to (R);
        \draw[thick,dotted] (N1) to (T);
        \draw[thick,dotted] (M) to (R);
        \draw[thick,dotted] (T2) to (R);
        \draw[thick,dotted] (R2) to (R);
        \draw[thick,dotted] (N2) to (T2);
        \draw[thick,-latex] (N1) to [bend left] (T);
        \draw[thick,-latex] (T) to (R);
        \draw[thick,-latex] (N2) to [bend left] (T2);
    \end{tikzpicture}

\end{frame}

\begin{frame}[fragile]{Avaliação da expressão \code{apl}{1-2+3}}

    \begin{tikzpicture}
        \node[opacity=0] at (0, 0) { . }; 
        \node[opacity=0] at (14, 7) { . }; 

        \node (E) at (4, 6) { $E$ };
        \node (T) at (1, 4.75) { $T.val = 1$ };
        \node (R) at (7, 4.75) { $R.h = 1$ };
        \node (N1) at (1, 3.5) { $\textbf{num}.val = 1$ };

        \node (M) at (5.5, 3.5) { $-$ };
        \node (T2) at (7, 3.5) { $T.val = 2$ };
        \node (R2) at (10, 3.5) { $R.h = -1$ };
        \node (N2) at (7, 2.25) { $\textbf{num}.val = 2$ };

        \draw[thick,dotted] (E) to (T);
        \draw[thick,dotted] (E) to (R);
        \draw[thick,dotted] (N1) to (T);
        \draw[thick,dotted] (M) to (R);
        \draw[thick,dotted] (T2) to (R);
        \draw[thick,dotted] (R2) to (R);
        \draw[thick,dotted] (N2) to (T2);
        \draw[thick,-latex] (N1) to [bend left] (T);
        \draw[thick,-latex] (T) to (R);
        \draw[thick,-latex] (N2) to [bend left] (T2);
        \draw[thick,-latex] (T2) to (R2);
    \end{tikzpicture}

\end{frame}

\begin{frame}[fragile]{Avaliação da expressão \code{apl}{1-2+3}}

    \begin{tikzpicture}
        \node[opacity=0] at (0, 0) { . }; 
        \node[opacity=0] at (14, 7) { . }; 

        \node (E) at (4, 6) { $E$ };
        \node (T) at (1, 4.75) { $T.val = 1$ };
        \node (R) at (7, 4.75) { $R.h = 1$ };
        \node (N1) at (1, 3.5) { $\textbf{num}.val = 1$ };

        \node (M) at (5.5, 3.5) { $-$ };
        \node (T2) at (7, 3.5) { $T.val = 2$ };
        \node (R2) at (10, 3.5) { $R.h = -1$ };
        \node (N2) at (7, 2.25) { $\textbf{num}.val = 2$ };

        \node (P) at (8.5, 2.25) { $+$ };
        \node (T3) at (10, 2.25) { $T$ };
        \node (R3) at (13, 2.25) { $R$ };


        \draw[thick,dotted] (E) to (T);
        \draw[thick,dotted] (E) to (R);
        \draw[thick,dotted] (N1) to (T);
        \draw[thick,dotted] (M) to (R);
        \draw[thick,dotted] (T2) to (R);
        \draw[thick,dotted] (R2) to (R);
        \draw[thick,dotted] (N2) to (T2);
        \draw[thick,dotted] (R2) to (P);
        \draw[thick,dotted] (R2) to (T3);
        \draw[thick,dotted] (R2) to (R3);
        \draw[thick,-latex] (N1) to [bend left] (T);
        \draw[thick,-latex] (T) to (R);
        \draw[thick,-latex] (N2) to [bend left] (T2);
        \draw[thick,-latex] (T2) to (R2);
    \end{tikzpicture}

\end{frame}

\begin{frame}[fragile]{Avaliação da expressão \code{apl}{1-2+3}}

    \begin{tikzpicture}
        \node[opacity=0] at (0, 0) { . }; 
        \node[opacity=0] at (14, 7) { . }; 

        \node (E) at (4, 6) { $E$ };
        \node (T) at (1, 4.75) { $T.val = 1$ };
        \node (R) at (7, 4.75) { $R.h = 1$ };
        \node (N1) at (1, 3.5) { $\textbf{num}.val = 1$ };

        \node (M) at (5.5, 3.5) { $-$ };
        \node (T2) at (7, 3.5) { $T.val = 2$ };
        \node (R2) at (10, 3.5) { $R.h = -1$ };
        \node (N2) at (7, 2.25) { $\textbf{num}.val = 2$ };

        \node (P) at (8.5, 2.25) { $+$ };
        \node (T3) at (10, 2.25) { $T$ };
        \node (R3) at (13, 2.25) { $R$ };
        \node (N3) at (10, 1) { $\textbf{num}$ };


        \draw[thick,dotted] (E) to (T);
        \draw[thick,dotted] (E) to (R);
        \draw[thick,dotted] (N1) to (T);
        \draw[thick,dotted] (M) to (R);
        \draw[thick,dotted] (T2) to (R);
        \draw[thick,dotted] (R2) to (R);
        \draw[thick,dotted] (N2) to (T2);
        \draw[thick,dotted] (R2) to (P);
        \draw[thick,dotted] (R2) to (T3);
        \draw[thick,dotted] (R2) to (R3);
        \draw[thick,dotted] (T3) to (N3);
        \draw[thick,-latex] (N1) to [bend left] (T);
        \draw[thick,-latex] (T) to (R);
        \draw[thick,-latex] (N2) to [bend left] (T2);
        \draw[thick,-latex] (T2) to (R2);
    \end{tikzpicture}

\end{frame}

\begin{frame}[fragile]{Avaliação da expressão \code{apl}{1-2+3}}

    \begin{tikzpicture}
        \node[opacity=0] at (0, 0) { . }; 
        \node[opacity=0] at (14, 7) { . }; 

        \node (E) at (4, 6) { $E$ };
        \node (T) at (1, 4.75) { $T.val = 1$ };
        \node (R) at (7, 4.75) { $R.h = 1$ };
        \node (N1) at (1, 3.5) { $\textbf{num}.val = 1$ };

        \node (M) at (5.5, 3.5) { $-$ };
        \node (T2) at (7, 3.5) { $T.val = 2$ };
        \node (R2) at (10, 3.5) { $R.h = -1$ };
        \node (N2) at (7, 2.25) { $\textbf{num}.val = 2$ };

        \node (P) at (8.5, 2.25) { $+$ };
        \node (T3) at (10, 2.25) { $T$ };
        \node (R3) at (13, 2.25) { $R$ };
        \node (N3) at (10, 1) { $\textbf{num}.val = 3$ };


        \draw[thick,dotted] (E) to (T);
        \draw[thick,dotted] (E) to (R);
        \draw[thick,dotted] (N1) to (T);
        \draw[thick,dotted] (M) to (R);
        \draw[thick,dotted] (T2) to (R);
        \draw[thick,dotted] (R2) to (R);
        \draw[thick,dotted] (N2) to (T2);
        \draw[thick,dotted] (R2) to (P);
        \draw[thick,dotted] (R2) to (T3);
        \draw[thick,dotted] (R2) to (R3);
        \draw[thick,dotted] (T3) to (N3);
        \draw[thick,-latex] (N1) to [bend left] (T);
        \draw[thick,-latex] (T) to (R);
        \draw[thick,-latex] (N2) to [bend left] (T2);
        \draw[thick,-latex] (T2) to (R2);
    \end{tikzpicture}

\end{frame}

\begin{frame}[fragile]{Avaliação da expressão \code{apl}{1-2+3}}

    \begin{tikzpicture}
        \node[opacity=0] at (0, 0) { . }; 
        \node[opacity=0] at (14, 7) { . }; 

        \node (E) at (4, 6) { $E$ };
        \node (T) at (1, 4.75) { $T.val = 1$ };
        \node (R) at (7, 4.75) { $R.h = 1$ };
        \node (N1) at (1, 3.5) { $\textbf{num}.val = 1$ };

        \node (M) at (5.5, 3.5) { $-$ };
        \node (T2) at (7, 3.5) { $T.val = 2$ };
        \node (R2) at (10, 3.5) { $R.h = -1$ };
        \node (N2) at (7, 2.25) { $\textbf{num}.val = 2$ };

        \node (P) at (8.5, 2.25) { $+$ };
        \node (T3) at (10, 2.25) { $T.val = 3$ };
        \node (R3) at (13, 2.25) { $R$ };
        \node (N3) at (10, 1) { $\textbf{num}.val = 3$ };


        \draw[thick,dotted] (E) to (T);
        \draw[thick,dotted] (E) to (R);
        \draw[thick,dotted] (N1) to (T);
        \draw[thick,dotted] (M) to (R);
        \draw[thick,dotted] (T2) to (R);
        \draw[thick,dotted] (R2) to (R);
        \draw[thick,dotted] (N2) to (T2);
        \draw[thick,dotted] (R2) to (P);
        \draw[thick,dotted] (R2) to (T3);
        \draw[thick,dotted] (R2) to (R3);
        \draw[thick,dotted] (T3) to (N3);
        \draw[thick,-latex] (N1) to [bend left] (T);
        \draw[thick,-latex] (T) to (R);
        \draw[thick,-latex] (N2) to [bend left] (T2);
        \draw[thick,-latex] (T2) to (R2);
        \draw[thick,-latex] (N3) to [bend left] (T3);
    \end{tikzpicture}

\end{frame}

\begin{frame}[fragile]{Avaliação da expressão \code{apl}{1-2+3}}

    \begin{tikzpicture}
        \node[opacity=0] at (0, 0) { . }; 
        \node[opacity=0] at (14, 7) { . }; 

        \node (E) at (4, 6) { $E$ };
        \node (T) at (1, 4.75) { $T.val = 1$ };
        \node (R) at (7, 4.75) { $R.h = 1$ };
        \node (N1) at (1, 3.5) { $\textbf{num}.val = 1$ };

        \node (M) at (5.5, 3.5) { $-$ };
        \node (T2) at (7, 3.5) { $T.val = 2$ };
        \node (R2) at (10, 3.5) { $R.h = -1$ };
        \node (N2) at (7, 2.25) { $\textbf{num}.val = 2$ };

        \node (P) at (8.5, 2.25) { $+$ };
        \node (T3) at (10, 2.25) { $T.val = 3$ };
        \node (R3) at (13, 2.25) { $R.h = 2$ };
        \node (N3) at (10, 1) { $\textbf{num}.val = 3$ };


        \draw[thick,dotted] (E) to (T);
        \draw[thick,dotted] (E) to (R);
        \draw[thick,dotted] (N1) to (T);
        \draw[thick,dotted] (M) to (R);
        \draw[thick,dotted] (T2) to (R);
        \draw[thick,dotted] (R2) to (R);
        \draw[thick,dotted] (N2) to (T2);
        \draw[thick,dotted] (R2) to (P);
        \draw[thick,dotted] (R2) to (T3);
        \draw[thick,dotted] (R2) to (R3);
        \draw[thick,dotted] (T3) to (N3);
        \draw[thick,-latex] (N1) to [bend left] (T);
        \draw[thick,-latex] (T) to (R);
        \draw[thick,-latex] (N2) to [bend left] (T2);
        \draw[thick,-latex] (T2) to (R2);
        \draw[thick,-latex] (N3) to [bend left] (T3);
        \draw[thick,-latex] (T3) to (R3);
    \end{tikzpicture}

\end{frame}

\begin{frame}[fragile]{Avaliação da expressão \code{apl}{1-2+3}}

    \begin{tikzpicture}
        \node[opacity=0] at (0, 0) { . }; 
        \node[opacity=0] at (14, 7) { . }; 

        \node (E) at (4, 6) { $E$ };
        \node (T) at (1, 4.75) { $T.val = 1$ };
        \node (R) at (7, 4.75) { $R.h = 1$ };
        \node (N1) at (1, 3.5) { $\textbf{num}.val = 1$ };

        \node (M) at (5.5, 3.5) { $-$ };
        \node (T2) at (7, 3.5) { $T.val = 2$ };
        \node (R2) at (10, 3.5) { $R.h = -1$ };
        \node (N2) at (7, 2.25) { $\textbf{num}.val = 2$ };

        \node (P) at (8.5, 2.25) { $+$ };
        \node (T3) at (10, 2.25) { $T.val = 3$ };
        \node (R3) at (13, 2.25) { $R.h = 2$ };
        \node (N3) at (10, 1) { $\textbf{num}.val = 3$ };
        \node (V) at (13, 1) { \code{apl}{∊} };


        \draw[thick,dotted] (E) to (T);
        \draw[thick,dotted] (E) to (R);
        \draw[thick,dotted] (N1) to (T);
        \draw[thick,dotted] (M) to (R);
        \draw[thick,dotted] (T2) to (R);
        \draw[thick,dotted] (R2) to (R);
        \draw[thick,dotted] (N2) to (T2);
        \draw[thick,dotted] (R2) to (P);
        \draw[thick,dotted] (R2) to (T3);
        \draw[thick,dotted] (R2) to (R3);
        \draw[thick,dotted] (T3) to (N3);
        \draw[thick,dotted] (R3) to (V);
        \draw[thick,-latex] (N1) to [bend left] (T);
        \draw[thick,-latex] (T) to (R);
        \draw[thick,-latex] (N2) to [bend left] (T2);
        \draw[thick,-latex] (T2) to (R2);
        \draw[thick,-latex] (N3) to [bend left] (T3);
        \draw[thick,-latex] (T3) to (R3);
    \end{tikzpicture}

\end{frame}

\begin{frame}[fragile]{Avaliação da expressão \code{apl}{1-2+3}}

    \begin{tikzpicture}
        \node[opacity=0] at (0, 0) { . }; 
        \node[opacity=0] at (14, 7) { . }; 

        \node (E) at (4, 6) { $E$ };
        \node (T) at (1, 4.75) { $T.val = 1$ };
        \node (R) at (7, 4.75) { $R.h = 1$ };
        \node (N1) at (1, 3.5) { $\textbf{num}.val = 1$ };

        \node (M) at (5.5, 3.5) { $-$ };
        \node (T2) at (7, 3.5) { $T.val = 2$ };
        \node (R2) at (10, 3.5) { $R.h = -1$ };
        \node (N2) at (7, 2.25) { $\textbf{num}.val = 2$ };

        \node (P) at (8.5, 2.25) { $+$ };
        \node (T3) at (10, 2.25) { $T.val = 3$ };
        \node (R3) at (13, 2.25) { ${R.s = 2}$ };
        \node (N3) at (10, 1) { $\textbf{num}.val = 3$ };
        \node (V) at (13, 1) { \code{apl}{∊} };


        \draw[thick,dotted] (E) to (T);
        \draw[thick,dotted] (E) to (R);
        \draw[thick,dotted] (N1) to (T);
        \draw[thick,dotted] (M) to (R);
        \draw[thick,dotted] (T2) to (R);
        \draw[thick,dotted] (R2) to (R);
        \draw[thick,dotted] (N2) to (T2);
        \draw[thick,dotted] (R2) to (P);
        \draw[thick,dotted] (R2) to (T3);
        \draw[thick,dotted] (R2) to (R3);
        \draw[thick,dotted] (T3) to (N3);
        \draw[thick,dotted] (R3) to (V);
        \draw[thick,-latex] (N1) to [bend left] (T);
        \draw[thick,-latex] (T) to (R);
        \draw[thick,-latex] (N2) to [bend left] (T2);
        \draw[thick,-latex] (T2) to (R2);
        \draw[thick,-latex] (N3) to [bend left] (T3);
        \draw[thick,-latex] (T3) to (R3);
        \draw[thick,-latex] (V) to [bend right] (R3);
    \end{tikzpicture}

\end{frame}

\begin{frame}[fragile]{Avaliação da expressão \code{apl}{1-2+3}}

    \begin{tikzpicture}
        \node[opacity=0] at (0, 0) { . }; 
        \node[opacity=0] at (14, 7) { . }; 

        \node (E) at (4, 6) { $E$ };
        \node (T) at (1, 4.75) { $T.val = 1$ };
        \node (R) at (7, 4.75) { $R.h = 1$ };
        \node (N1) at (1, 3.5) { $\textbf{num}.val = 1$ };

        \node (M) at (5.5, 3.5) { $-$ };
        \node (T2) at (7, 3.5) { $T.val = 2$ };
        \node (R2) at (10, 3.5) { $R.s = 2$ };
        \node (N2) at (7, 2.25) { $\textbf{num}.val = 2$ };

        \node (P) at (8.5, 2.25) { $+$ };
        \node (T3) at (10, 2.25) { $T.val = 3$ };
        \node (R3) at (13, 2.25) { ${R.s = 2}$ };
        \node (N3) at (10, 1) { $\textbf{num}.val = 3$ };
        \node (V) at (13, 1) { \code{apl}{∊} };


        \draw[thick,dotted] (E) to (T);
        \draw[thick,dotted] (E) to (R);
        \draw[thick,dotted] (N1) to (T);
        \draw[thick,dotted] (M) to (R);
        \draw[thick,dotted] (T2) to (R);
        \draw[thick,dotted] (R2) to (R);
        \draw[thick,dotted] (N2) to (T2);
        \draw[thick,dotted] (R2) to (P);
        \draw[thick,dotted] (R2) to (T3);
        \draw[thick,dotted] (R2) to (R3);
        \draw[thick,dotted] (T3) to (N3);
        \draw[thick,dotted] (R3) to (V);
        \draw[thick,-latex] (N1) to [bend left] (T);
        \draw[thick,-latex] (T) to (R);
        \draw[thick,-latex] (N2) to [bend left] (T2);
        \draw[thick,-latex] (T2) to (R2);
        \draw[thick,-latex] (N3) to [bend left] (T3);
        \draw[thick,-latex] (T3) to (R3);
        \draw[thick,-latex] (V) to [bend right] (R3);
        \draw[thick,-latex] (R3) to [bend right] (R2);
    \end{tikzpicture}

\end{frame}

\begin{frame}[fragile]{Avaliação da expressão \code{apl}{1-2+3}}

    \begin{tikzpicture}
        \node[opacity=0] at (0, 0) { . }; 
        \node[opacity=0] at (14, 7) { . }; 

        \node (E) at (4, 6) { $E$ };
        \node (T) at (1, 4.75) { $T.val = 1$ };
        \node (R) at (7, 4.75) { $R.s = 2$ };
        \node (N1) at (1, 3.5) { $\textbf{num}.val = 1$ };

        \node (M) at (5.5, 3.5) { $-$ };
        \node (T2) at (7, 3.5) { $T.val = 2$ };
        \node (R2) at (10, 3.5) { $R.s = 2$ };
        \node (N2) at (7, 2.25) { $\textbf{num}.val = 2$ };

        \node (P) at (8.5, 2.25) { $+$ };
        \node (T3) at (10, 2.25) { $T.val = 3$ };
        \node (R3) at (13, 2.25) { ${R.s = 2}$ };
        \node (N3) at (10, 1) { $\textbf{num}.val = 3$ };
        \node (V) at (13, 1) { \code{apl}{∊} };


        \draw[thick,dotted] (E) to (T);
        \draw[thick,dotted] (E) to (R);
        \draw[thick,dotted] (N1) to (T);
        \draw[thick,dotted] (M) to (R);
        \draw[thick,dotted] (T2) to (R);
        \draw[thick,dotted] (R2) to (R);
        \draw[thick,dotted] (N2) to (T2);
        \draw[thick,dotted] (R2) to (P);
        \draw[thick,dotted] (R2) to (T3);
        \draw[thick,dotted] (R2) to (R3);
        \draw[thick,dotted] (T3) to (N3);
        \draw[thick,dotted] (R3) to (V);
        \draw[thick,-latex] (N1) to [bend left] (T);
        \draw[thick,-latex] (T) to (R);
        \draw[thick,-latex] (N2) to [bend left] (T2);
        \draw[thick,-latex] (T2) to (R2);
        \draw[thick,-latex] (N3) to [bend left] (T3);
        \draw[thick,-latex] (T3) to (R3);
        \draw[thick,-latex] (V) to [bend right] (R3);
        \draw[thick,-latex] (R3) to [bend right] (R2);
        \draw[thick,-latex] (R2) to [bend right] (R);
    \end{tikzpicture}

\end{frame}

\begin{frame}[fragile]{Avaliação da expressão \code{apl}{1-2+3}}

    \begin{tikzpicture}
        \node[opacity=0] at (0, 0) { . }; 
        \node[opacity=0] at (14, 7) { . }; 

        \node (E) at (4, 6) { $E.val = 2$ };
        \node (T) at (1, 4.75) { $T.val = 1$ };
        \node (R) at (7, 4.75) { $R.s = 2$ };
        \node (N1) at (1, 3.5) { $\textbf{num}.val = 1$ };

        \node (M) at (5.5, 3.5) { $-$ };
        \node (T2) at (7, 3.5) { $T.val = 2$ };
        \node (R2) at (10, 3.5) { $R.s = 2$ };
        \node (N2) at (7, 2.25) { $\textbf{num}.val = 2$ };

        \node (P) at (8.5, 2.25) { $+$ };
        \node (T3) at (10, 2.25) { $T.val = 3$ };
        \node (R3) at (13, 2.25) { ${R.s = 2}$ };
        \node (N3) at (10, 1) { $\textbf{num}.val = 3$ };
        \node (V) at (13, 1) { \code{apl}{∊} };


        \draw[thick,dotted] (E) to (T);
        \draw[thick,dotted] (E) to (R);
        \draw[thick,dotted] (N1) to (T);
        \draw[thick,dotted] (M) to (R);
        \draw[thick,dotted] (T2) to (R);
        \draw[thick,dotted] (R2) to (R);
        \draw[thick,dotted] (N2) to (T2);
        \draw[thick,dotted] (R2) to (P);
        \draw[thick,dotted] (R2) to (T3);
        \draw[thick,dotted] (R2) to (R3);
        \draw[thick,dotted] (T3) to (N3);
        \draw[thick,dotted] (R3) to (V);
        \draw[thick,-latex] (N1) to [bend left] (T);
        \draw[thick,-latex] (T) to (R);
        \draw[thick,-latex] (N2) to [bend left] (T2);
        \draw[thick,-latex] (T2) to (R2);
        \draw[thick,-latex] (N3) to [bend left] (T3);
        \draw[thick,-latex] (T3) to (R3);
        \draw[thick,-latex] (V) to [bend right] (R3);
        \draw[thick,-latex] (R3) to [bend right] (R2);
        \draw[thick,-latex] (R2) to [bend right] (R);
        \draw[thick,-latex] (R) to [bend right] (E);
    \end{tikzpicture}

\end{frame}

